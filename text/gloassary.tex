\newglossaryentry{Metrik}
{
   name={Metrik},
   description={ist eine Eigenschaft oder der Wert dieser Eigenschaft und wird f\"ur Methoden, Typen oder Namespaces bestimmt}
}

\newglossaryentry{CC}
{
   name={Zyklomatische Komplexit\"at},
   description={ist eine \gls{Metrik} von Methoden und bezeichnet die Anzahl der unabh\"angigen M\"oglichkeiten eine Methode zu durchlaufen}
}

\newglossaryentry{ML}
{
   name={Methodenl\"ange},
   description={ist eine \gls{Metrik} von Methoden und bezeichnet im Kontext von Usus.NET die Anzahl der Code-Zeilen die Logik enthalten}
}

\newglossaryentry{CS}
{
   name={Klassengr\"o\ss e},
   description={ist eine \gls{Metrik} von Klassen und Interfaces und bezeichnet im Kontext von Usus.NET die Anzahl der Methoden eines Typen}
}

\newglossaryentry{CCD}
{
   name={Kumulierte Komponentenabh\"angigkeit},
   description={ist eine \gls{Metrik} von Klassen und Interfaces und bezeichnet die Anzahl anderer Klassen und Interfaces, von denen die Klasse oder das Interface direkt und indirekt abh\"angig ist}
}

\newglossaryentry{ACD}
{
   name={Durchschnittliche Kumulierte Komponentenabh\"angigkeit},
   description={ist eine \gls{Metrik} einer Menge an Klassen und Interfaces und bezeichnet den Durchschnitt aller \glslink{CCD}{kumulierter Komponentenabh\"angigkeiten} der enthaltenen Klassen und Interfaces}
}

\newglossaryentry{ACS}
{
   name={Durchschnittliche Klassengr\"o\ss e},
   description={ist eine \gls{Metrik} einer Menge an Klassen und Interfaces und bezeichnet den Durchschnitt aller \glslink{CS}{Klassengr\"o\ss en} der enthaltenen Klassen und Interfaces}
}

\newglossaryentry{ACC}
{
   name={Durchschnittliche Zyklomatische Komplexit\"at},
   description={ist eine \gls{Metrik} einer Menge an Klassen und Interfaces und bezeichnet den Durchschnitt aller \glslink{CC}{zyklomatischer Komplexit\"aten} der Methoden aller enthaltenen Klassen und Interfaces}
}

\newglossaryentry{AML}
{
   name={Durchschnittliche Methodenl\"ange},
   description={ist eine \gls{Metrik} einer Menge an Klassen und Interfaces und bezeichnet den Durchschnitt aller \glslink{ML}{Methodenl\"angen} der Methoden aller enthaltenen Klassen und Interfaces}
}

\newglossaryentry{NSPF}
{
   name={Nicht-statische \"offentliche Felder},
   description={ist eine \gls{Metrik} einer Menge an Klassen und bezeichnet im Kontext von Usus.NET die Anzahl aller Klassen, die mindestens ein \"offentliches Feld haben, das nicht statisch ist}
}

\newglossaryentry{NCD}
{
   name={Namespaces mit zyklischen Abh\"angigkeiten},
   description={ist eine \gls{Metrik} einer Menge an Namespaces und bezeichnet im Kontext von Usus.NET die Anzahl aller Namespaces, die in einem Zyklus sind}
}

\newglossaryentry{CCI}
{
   name={Common Compiler Infrastructure},
   description={ist eine umfangreiche Bibliothek die es erm\"oglicht \glslink{SCA}{statische Code-Analysen} durchzuf\"uhren. Sie besteht aus den beiden Komponenten \gls{CCIMetadata} und \gls{CCIAst}}
}

\newglossaryentry{CCIAst}
{
   name={CCI Code and AST Components},
   description={ist eine von Mircosoft ver\"offentlichte Komponente, die es erlaubt Methoden zu dekompilieren und einen \glslink{Ast}{abstrakten Syntax-Baum} zu erstellen}
}

\newglossaryentry{CCIMetadata}
{
   name={CCI Metadata},
   description={ist eine von Mircosoft ver\"offentlichte Komponente, die es erlaubt \glslink{ASM}{Assemblies} zu analysieren um Klassen-, Interface- und Methodendaten zu ermitteln}
}

\newglossaryentry{Ast}
{
   name={Abstrakter Syntax-Baum},
   description={ist eine syntaktische Repr\"asentation von Methoden die f\"ur jede Anweisung einen Knoten enth\"alt}
}

\newglossaryentry{SCA}
{
   name={Statische Code-Analyse},
   description={bezeichnet die Analyse eines Software-Programms, ohne das dieses ausgef\"uhrt werden muss. Als Ergebnis wird ein Bericht, beispielsweise �ber \glslink{Metrik}{Metriken}, erstellt}
}

\newglossaryentry{ASM}
{
   name={Assembly},
   description={ist eine Datei, die ein .NET-Compiler aus einem Visual Studio-Projekt erzeugt. Die Datei ist entweder eine exe- oder eine dll-Datei. Synonym wird eine Assembly auch als PE-Datei (Portable Executable) bezeichnet}
}

\newglossaryentry{PDB}
{
   name={Program Database},
   description={ist die pdb-Datei, die ein .NET-Compiler neben der \gls{ASM} erzeugt. Diese Datei enth�lt Dokument- und Zeileninformationen der Typen und Methoden in der entsprechenden \glslink{ASM}{PE-Datei}}
}