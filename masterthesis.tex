\documentclass[
a4paper, 							% Papierformat
%10pt,								% Schriftgr��e (12pt, 11pt (Standard))
%twoside,							% Doppelseiten
titlepage,						% Titelei auf eigener Seite
normalheadings,				% �berschriften etwas kleiner (smallheadings)
%idxtotoc,						% Index im Inhaltsverzeichnis
%liststotoc,					% Abb.- und Tab.verzeichnis im Inhalt
%bibtotoc,						% Literaturverzeichnis im Inhalt
%leqno,   						% Nummerierung von Gleichungen links
%fleqn,								% Ausgabe von Gleichungen linksb�ndig
%draft								% �berlangen Zeilen in Ausgabe gekennzeichnet
]
{scrreprt}
\usepackage[ngerman]{babel}
\usepackage[isolatin]{inputenc}
\usepackage{graphicx}
\usepackage{url}
\usepackage{tipa}
\usepackage[left=2.5cm,right=2.5cm,top=2.5cm,bottom=2.5cm,includeheadfoot]{geometry}

\begin{document}



\thispagestyle{empty}
\begin{titlepage}
\begin{figure}[t]
	\centering
  \includegraphics[width=80mm]{images/HsKaLogoKlein.png}
	\vspace{2.5cm}
\end{figure}
\begin{center}
\title{Master-Thesis}
\textbf{\huge{Master-Thesis}} \\[0.5cm]
\textbf{Visual Studio Erweiterung zur statischen Code-Analyse} \\[4cm]
\textbf{andrena objetcs ag} \\[0.25cm]
\author{Manuel Naujoks} Manuel Naujoks\\[2.5cm]
Betreut durch \\[0.25cm] 
Prof. Dr. Thomas Fuch� \\[2.5cm]
Bischweier, den \today
\end{center}
\end{titlepage}



\thispagestyle{empty}
\begin{center}\textbf{\large Erkl�rung}\\[1cm]\end{center}
Hiermit versichere ich, dass ich die vorliegende Arbeit selbstst�ndig verfasst und keine anderen als die angegebenen Quellen und Hilfsmittel benutzt habe, dass alle Stellen der Arbeit, die w�rtlich oder sinngem�� aus anderen Quellen �bernommen wurden, als solche kenntlich gemacht sind und dass die Arbeit in gleicher oder �hnlicher Form noch keiner Pr�fungsbeh�rde vorgelegt wurde.
\\[4\baselineskip]
Bischweier, den \today \\
Manuel Naujoks
\newpage



\thispagestyle{empty}
\textbf{\large Zusammenfassung}\\[1cm]
Im Rahmen dieser Master-Thesis soll eine Visual Studio Erweiterung entwickelt werden, die direktes Entwickler-Feedback anhand von Software-Metriken geben kann. Dabei dient das Plugin Usus f�r Eclipse als Vorlage, welches bereits existiert. Die zu erstellende Erweiterung soll eine statische Code-Analyse von .NET-Projekten in Visual Stuido 2010 durchf�hren und f�r andrena relevante Code-Metriken berechnen k�nnen.
\newline
Die zu entwickelnde Erweiterung soll genutzt werden k�nnen, um Software-Entwickler aktiv zu unterst�tzen "Clean code" zu schreiben. Eine �hnliche L�sung zu Microsofts Achievements Extension mit Achievements in Bezug auf clean code best practices w�re denkbar. Dazu soll eine Evaluierung anhand von Beispielaufgaben aus einem andrena-Kurs zum Thema Refaktorisierung bearbeitet und die Ver�nderung in den Metriken entsprechend erfasst und dokumentiert werden.
\newline
Weiterhin soll in einer Metrik-Analyse nach Heuristiken oder Regeln mit statischer Signifikanz gesucht werden, die eventuell guten von schlechtem Code unterscheiden k�nnen. Lassen sich hier Muster beziehungsweise Strukturen aufzeigen? Ein Indiz hierf�r ist, dass die Metriken oft einer Exponentialverteilung folgen. Dabei soll untersucht werden, ob und wenn m�glich wie sich dies auf den Software Qualit�ts Index (SQI) von andrena abbilden l�sst.
\newpage



\thispagestyle{empty}
\textbf{\large Abstract}\\[1cm]
Objective of this master thesis is the development of a Visual Studio Extension that is capable of providing direct development feedback based on software metrics. The Eclipse plugin Usus, which already exists, is going to be used as orientation. The extension that is developed shall be able to perform static code analysis of .NET projects in Visual Studio 2010 in order to calculate the code metrics that are relevant for andrena.
\newline
As far as feedback is concerned, the extension shall be able to actively support developers to write "clean code". A similar solution to Microsofts Achievements Extension could likely be found with achievements based on common clean code best practices. Therefore an evaluation with sample exercises of an andrena course on the topic of refactoring is done and the variation in the metrics is detected and documented.
\newline
Another metric analysis is performed in order to find heuristics or rules with static significance, which might be able to distinguish good code from bad code. Are there detectable patterns? One thing might be that metrics often follow an exponential distribution. An analysis shall show whether it is possible and if yes, how this can be mapped to the Software Quality Index (SQI) of andrena.
\newpage




\begingroup
\addtocontents{toc}{\protect\thispagestyle{empty}}
\pagestyle{empty}
\tableofcontents
\newpage
\setcounter{page}{1}
\endgroup



\chapter{Einf�hrung}
todo \cite{Wikipedia2011} todo test test

\bibliography{masterthesis}
\bibliographystyle{alpha}

\end{document}
